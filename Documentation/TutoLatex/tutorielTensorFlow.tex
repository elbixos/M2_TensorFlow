\documentclass[a4paper,11pt]{book}
\usepackage[francais]{babel}
\usepackage[table]{xcolor}
\usepackage{amsmath}
\usepackage[utf8]{inputenc}
\usepackage{textcomp}
\usepackage{gensymb}
%\usepackage[francais]{babel}
\usepackage[T1]{fontenc}
\usepackage{makeidx}
\usepackage{graphicx}
\usepackage{enumitem}
\usepackage{float}
\usepackage{relsize}
\usepackage{amsmath,amsfonts,amssymb}
\usepackage{multirow}
\usepackage{layout}
\usepackage{amsthm}
\usepackage{lmodern}
\usepackage{fancyhdr}
\usepackage{enumitem}
\usepackage{geometry}
\usepackage{listings}
\definecolor{darkWhite}{rgb}{0.94,0.94,0.94}
\lstset{
aboveskip=3mm,
belowskip=-2mm,
backgroundcolor=\color{darkWhite},
basicstyle=\footnotesize,
breakatwhitespace=false,
breaklines=true,
captionpos=b,
commentstyle=\color{red},
deletekeywords={...},
escapeinside={\%*}{*)},
extendedchars=true,
framexleftmargin=16pt,
framextopmargin=3pt,
framexbottommargin=6pt,
frame=tb,
keepspaces=true,
keywordstyle=\color{blue},
language=Python,
literate=
{²}{{\textsuperscript{2}}}1
{⁴}{{\textsuperscript{4}}}1
{⁶}{{\textsuperscript{6}}}1
{⁸}{{\textsuperscript{8}}}1
{€}{{\euro{}}}1
{é}{{\'e}}1
{è}{{\`{e}}}1
{ê}{{\^{e}}}1
{ë}{{\¨{e}}}1
{É}{{\'{E}}}1
{Ê}{{\^{E}}}1
{û}{{\^{u}}}1
{ù}{{\`{u}}}1
{â}{{\^{a}}}1
{à}{{\`{a}}}1
{á}{{\'{a}}}1
{ã}{{\~{a}}}1
{Á}{{\'{A}}}1
{Â}{{\^{A}}}1
{Ã}{{\~{A}}}1
{ç}{{\c{c}}}1
{Ç}{{\c{C}}}1
{õ}{{\~{o}}}1
{ó}{{\'{o}}}1
{ô}{{\^{o}}}1
{Õ}{{\~{O}}}1
{Ó}{{\'{O}}}1
{Ô}{{\^{O}}}1
{î}{{\^{i}}}1
{Î}{{\^{I}}}1
{í}{{\'{i}}}1
{Í}{{\~{Í}}}1,
morekeywords={*,...},
numbers=none,
numbersep=10pt,
numberstyle=\tiny\color{black},
rulecolor=\color{black},
showspaces=false,
showstringspaces=false,
showtabs=false,
stepnumber=1,
stringstyle=\color{gray},
tabsize=4,
title=\lstname,
}
\geometry{hmargin=2.2cm,vmargin=2.9cm}

 
\pagestyle{fancy}
\fancyhf{}
\fancyhead[L]{\leftmark}
\fancyfoot[C]{\thepage}
    
\theoremstyle{theo}
%\newtheorem{thm}{Theorem}

\newtheorem{theorem}{Théorème}[section]
\newtheorem{corollary}{Corollaire}[theorem]
\newtheorem{lemma}[theorem]{Lemme}
\newtheorem{prop}{Proposition}
	

\newcommand*{\QEDA}{\hfill\ensuremath{\blacksquare}}%
\newcommand*{\QEDB}{\hfill\ensuremath{\square}}%
\newcommand*{\R}{{\rm I\!R}}
\newcommand*{\Hi}{\mathcal{H}}

\newcommand\vbar[1]{\vrule\begin{tabular}[t]{l}#1\end{tabular}}
\newcommand{\bproof}{\begin{itemize} \item {\bf D\'emonstration.
\hspace{0.2cm}}}
\newcommand{\eproof}{\end{itemize}}
\usepackage[memoire]{PageDeGarde}
\def\CadreBleuhpos{500}

\def\entetehpos{-50}
\def\entetevpos{540}


%===================================
%  Fin renseignements a completer
%===================================


% ==================================================================
%               FIN TITLE PAGE 
% ==================================================================
\setlength{\headheight}{15.35403pt}
\begin{document}
\let\cleardoublepage\clearpage
\title{ \usefont{T1}{ptm}{m}{n} Tutoriel TensorFlow}

\annee{2017-2018}

\Auteur{IKNI Layachi}{}

\Encadrant{Pag\'e Vincent}{MCF à l'Université des Antilles}


\maketitle

\newpage
\section*{Remerciements}
\bigskip
\pagestyle{plain}
\bigskip
\newpage 
\tableofcontents
\newpage
\chapter{Introduction }
\
Il s'agit d'\'etudier......
\textbf{Remarque} : 
\begin{itemize}
\item Le système est d'ordre 2 ....$\nabla^2\phi(x(.))$. 
\item Par contre, le second système est d'ordre 3.
\end{itemize}
\par

\begin{lstlisting}
n=10;

u = zeros(n,1);
u(1:2) = [1;1];

for i = 1 : n-2;
  u(i+2) = u(i+1) + u(i);
end

disp(u)
\end{lstlisting}


\bigskip
\thispagestyle{empty}


\newpage


\newpage
\chapter{\'Etude analytique}
\pagestyle{fancy}
\fancyhf{}
\fancyhead[L]{\leftmark}
\fancyfoot[C]{\thepage}
\section{Équivalence des systèmes considérés}
Considérons les équations (1.1)-(1.2) et supposons que les constantes $a,b,\alpha$ et $\beta$ sont telles que 
\begin{equation} \beta \neq 0, \quad b=1/\beta, \quad a=\alpha-1/\beta \end{equation}
\begin{theorem} : Supposons $\Phi$ deux fois différentiable, $\Psi$ différentiable sur $\Hi$ et $(2.1)$ vérifié. Soient $(x_0,v_0,y_0)\in\Hi^3$ tels que $v_0+\beta\nabla\Phi(x_0) + ax_0+by_0=0$, alors les assertions suivantes sont équivalentes 
\begin{enumerate}[label=(\roman*)]
\item $x$ est une solution de $(1.1)$ deux fois différentiable avec les conditions initiales $x(0)=x_0$ et $\dot{x}(0)=v_0$.
\item Il existe une fonction $y$ telle que $(x,y)$ soit une solution différentiable de $(1.2)$ avec les conditions initiales $x(0)=x_0$ et $y(0)=y_0$
\end{enumerate}
\end{theorem}
\noindent\textbf{Démonstration} : \quad $(i)\implies (ii)$ Considérons la fonction $y$ définie par la première équation de $(1.2)$.
\begin{equation}
 \dot{x}+\beta\nabla\Phi(x) + \left(\alpha-\frac{1}{\beta}\right)x + \frac{1}{\beta}y=0
\end{equation}
\section{Existence et unicité des solutions }
\subsection{Cas où le potentiel $\Phi$ est régulier}
On s'intéresse ici à des résultats d'existence et d'unicité des équations $(1.1)$ et $(1.2)$. 
\begin{equation} \left\{
\begin{array}{l l l}
\dot{x}(t) + \nabla\Phi(x(t)) + ax(t) + by(t) &=& 0\\
\dot{y}(t) - \nabla\Psi(x(t)) + ax(t) + by(t) &=& 0 \\
x(0)=x_0, ~y(0)=y_0 
\end{array}
\right. 
\end{equation}
où $(x_0,y_0) \in \Hi^2$ et $a,b$ sont des réels	tels que
\medskip
\begin{description}
\item[(CP)] $b\geq 0, ~a+b\geq 0$
\end{description}
\medskip
et les fonction $\Phi$ et $\Psi$ sont supposées remplir les conditions suivantes
\medskip
\begin{description}
\item[(A1)] $\Psi : \Hi \to \R$ est différentiable, et à gradient lipschitzien sur des parties bornées de $\Hi$.
\medskip
\item[(A2)] $\Phi : \Hi \to \R$ est différentiable, convexe et à gradient lipschitzien sur les parties bornées de $\Hi$.
\medskip
\item[(A3)] $\Theta = \Phi+\Psi$ est minoré	sur $\Hi$.
\end{description}
\bigskip
\textbf{Remarque} : A chaque solution $(x,y)$ de $(2.3)$ on associe la fonction \og énergie \fg ~$E$ définie pour $t\geq 0$ par 
$$ E(t) = b\Theta(x(t))+\frac{1}{2}||\dot{x}(t)||^2 $$
L'existence des solutions découle alors de la décroissance de cette fonction \og énergie \fg ~$E$.

\begin{lemma}
Supposons que les conditions \textbf{\upshape(A1), (A2)} et \textbf{\upshape (CP)} soient satisfaites, supposons de plus que $(x,y)$ soit une solution de $(2.3)$ définie sur un intervalle non vide $[0,T_m[$. Alors $E$ est une fonction décroissante du temps et plus précisément, on a $ \forall s,t \in {\rm I\!R}, ~ 0\leq s \leq t < T_m$
\begin{equation}
\frac{1}{2}||\dot{x}(t)||^2 +  b\Theta(x(t))+(a+b)\int_{s}^t||\dot{x}(\tau)||^2 d\tau \leq \frac{1}{2}||\dot{x}(s)||^2 +  b\Theta(x(s))
\end{equation}
\end{lemma}

\begin{theorem}
Supposons que les conditions \textbf{\upshape(A1), (A2), (A3)} et \textbf{\upshape(CP)} soient satisfaites alors le problème $(2.3)$ admet une unique solution globale $(x,y)$ définie sur $[0;+\infty[$ et on a les propriétés suivantes :
\begin{description}
\item[(r1)] $x,y\in C^1([0;+\infty[;\Hi)$.
\medskip
\item[(r2)] $\dot{x}\in L^{\infty}([0;+\infty[;\Hi)$ et si de plus $a+b>0$ alors $\dot{x}\in L^2([0;+\infty[;\Hi)$
\medskip
\item[(r3)] L'énergie $E(t) = (1/2)|\dot{x}(t)|^2+b\Theta(x(t))$ est une fonction décroissante du temps.
\end{description}
\end{theorem}


\subsection{Cas où le potentiel $\phi$ n'est pas régulier}

\subsubsection{L'approximation de Moreau-Yosida}
Cette fois ci, au lieu d'étudier le problème $(2.3)$ , nous considérons le système suivant
\begin{equation} \left\{
\begin{array}{l l l}
\dot{x}(t) + \partial\Phi(x(t)) + ax(t) + by(t) &\ni& 0 \quad \text{pour presque tout} \quad t>0\\
\dot{y}(t) - \nabla\Psi(x(t)) + ax(t) + by(t) &=& 0 \quad \text{pour} \quad t> 0 \\
x(0)=x_0, ~y(0)=y_0 
\end{array}
\right. 
\end{equation}
 
$$ \dot{F}(t) = -b\langle \nabla\Theta(x(t)),x(t)-q\rangle $$
D'où $F$ décroissante pour $q\in S:=(\nabla\Theta)^{-1}(0)$
car la fonction $t\mapsto  \langle \Theta(x(t)),x(t)-q\rangle \in L^1([0,+\infty[)$ et est lipschitzienne, (en effet, $x$ est bornée , $\nabla\Theta$ est lipschitzienne et on conclut avec l'inégalité de Cauchy-Schwarz).
L'équation $(3.3)$ peut s'écrire sous la forme
$$ \lim_{t\to+\infty} \langle \nabla\Phi(x(t))-\nabla\Phi(q),x(t)-q \rangle - \langle \nabla\Psi(x(t))-\nabla\Psi(q),x(t)-q \rangle = 0$$
De plus, comme $\Phi$ et $\Psi$ sont supposés convexe, on a 
$$ \langle \nabla\Phi(x(t))-\nabla\Phi(q),x(t)-q \rangle \geq 0 \quad \text{et} \quad \nabla\Psi(x(t))-\nabla\Psi(q),x(t)-q \rangle \geq 0$$
D'où 
\begin{equation}
\displaystyle\lim_{t\to+\infty} \nabla\Phi(x(t))-\nabla\Phi(q),x(t)-q \rangle = 0
\end{equation}
En utilisant la convexité de $\Phi$, on a pour $z,q\in \Hi$
$$  \Phi(q)\geq \Phi(z) + \langle\nabla\Phi(z),q-z\rangle \qquad \text{et} \qquad  \Phi(z)\geq \Phi(q) + \langle\nabla\Phi(q),z-q\rangle $$
Ce qui nous donne $N(z,q) \geq 0$ et $\langle -\nabla\Phi(q),z-q,\rangle\geq \Phi(q)-\Phi(z)$, d'où finalement
$$ 0 \leq N(z,q) \leq \langle \nabla\Phi(z)-\nabla\Phi(q),z-q\rangle$$
En particulier, pour $q\in S$ et en utilisant $(3.4)$, on en déduit que $\lim_{t\to+\infty} N(x(t),q)=0$, puis comme $\overline{w},\overline{v}\in S$, on a, en utilisant $(3.2)$, 
$$\lim_{t\to+\infty} R_n = (a+b)||\overline{v}-\overline{w}||^2=0$$
d'où finalement $\overline{w}=\overline{v}$. \QEDA

\chapter*{Conclusion}
Ce T.E.R....... 
\newpage
\begin{thebibliography}{9}
\bibitem{seco}
Hédy Attouch, Paul-\'Emile Maingé, Patrick Redont, \emph{A second-order differential system with hessian-driven damping; application to non-elastic shock laws}, 2013
\bibitem{attou}
Hédy Attouch, \emph{Variational Convergence for Functions and Operators}, 1986
\bibitem{Brezis}
Haim Brézis, Analyse fontionnelle, 1987
\bibitem{Brezis2}
Haim Brézis, \emph{Opérateurs maximaux monotones et semie-groupes de contractions dans les espaces de Hilbert}, 1973
\bibitem{Nawfal}
Nawfal El Hage Hassan, \emph{Topologie générale et espaces normés}, 2011
\end{thebibliography}
\addcontentsline{toc}{chapter}{Références}
\bibliographystyle{alpha}

\end{document}
