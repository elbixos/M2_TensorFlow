\documentclass[a4paper,11pt]{book}
\usepackage[francais]{babel}
\usepackage[table]{xcolor}
\usepackage{amsmath}
\usepackage[utf8]{inputenc}
\usepackage{textcomp}
\usepackage{gensymb}
%\usepackage[francais]{babel}
\usepackage[T1]{fontenc}
\usepackage{makeidx}
\usepackage{graphicx}
\usepackage{enumitem}
\usepackage{float}
\usepackage{relsize}
\usepackage{amsmath,amsfonts,amssymb}
\usepackage{multirow}
\usepackage{layout}
\usepackage{amsthm}
\usepackage{lmodern}
\usepackage{fancyhdr}
\usepackage{enumitem}
\usepackage{geometry}
\usepackage{listings}
\definecolor{darkWhite}{rgb}{0.94,0.94,0.94}
\lstset{
aboveskip=3mm,
belowskip=-2mm,
backgroundcolor=\color{darkWhite},
basicstyle=\footnotesize,
breakatwhitespace=false,
breaklines=true,
captionpos=b,
commentstyle=\color{red},
deletekeywords={...},
escapeinside={\%*}{*)},
extendedchars=true,
framexleftmargin=16pt,
framextopmargin=3pt,
framexbottommargin=6pt,
frame=tb,
keepspaces=true,
keywordstyle=\color{blue},
language=Python,
literate=
{²}{{\textsuperscript{2}}}1
{⁴}{{\textsuperscript{4}}}1
{⁶}{{\textsuperscript{6}}}1
{⁸}{{\textsuperscript{8}}}1
{€}{{\euro{}}}1
{é}{{\'e}}1
{è}{{\`{e}}}1
{ê}{{\^{e}}}1
{ë}{{\¨{e}}}1
{É}{{\'{E}}}1
{Ê}{{\^{E}}}1
{û}{{\^{u}}}1
{ù}{{\`{u}}}1
{â}{{\^{a}}}1
{à}{{\`{a}}}1
{á}{{\'{a}}}1
{ã}{{\~{a}}}1
{Á}{{\'{A}}}1
{Â}{{\^{A}}}1
{Ã}{{\~{A}}}1
{ç}{{\c{c}}}1
{Ç}{{\c{C}}}1
{õ}{{\~{o}}}1
{ó}{{\'{o}}}1
{ô}{{\^{o}}}1
{Õ}{{\~{O}}}1
{Ó}{{\'{O}}}1
{Ô}{{\^{O}}}1
{î}{{\^{i}}}1
{Î}{{\^{I}}}1
{í}{{\'{i}}}1
{Í}{{\~{Í}}}1,
morekeywords={*,...},
numbers=none,
numbersep=10pt,
numberstyle=\tiny\color{black},
rulecolor=\color{black},
showspaces=false,
showstringspaces=false,
showtabs=false,
stepnumber=1,
stringstyle=\color{gray},
tabsize=4,
title=\lstname,
}
\geometry{hmargin=2.2cm,vmargin=2.9cm}

 
\pagestyle{fancy}
\fancyhf{}
\fancyhead[L]{\leftmark}
\fancyfoot[C]{\thepage}
    
\theoremstyle{theo}
%\newtheorem{thm}{Theorem}

\newtheorem{theorem}{Théorème}[section]
\newtheorem{corollary}{Corollaire}[theorem]
\newtheorem{lemma}[theorem]{Lemme}
\newtheorem{prop}{Proposition}
	

\newcommand*{\QEDA}{\hfill\ensuremath{\blacksquare}}%
\newcommand*{\QEDB}{\hfill\ensuremath{\square}}%
\newcommand*{\R}{{\rm I\!R}}
\newcommand*{\Hi}{\mathcal{H}}

\newcommand\vbar[1]{\vrule\begin{tabular}[t]{l}#1\end{tabular}}
\newcommand{\bproof}{\begin{itemize} \item {\bf D\'emonstration.
\hspace{0.2cm}}}
\newcommand{\eproof}{\end{itemize}}
\usepackage[memoire]{PageDeGarde}
\def\CadreBleuhpos{500}

\def\entetehpos{-50}
\def\entetevpos{540}


%===================================
%  Fin renseignements a completer
%===================================


% ==================================================================
%               FIN TITLE PAGE 
% ==================================================================
\setlength{\headheight}{15.35403pt}
\begin{document}
\let\cleardoublepage\clearpage
\title{ \usefont{T1}{ptm}{m}{n} Tutoriel TensorFlow}

\annee{2017-2018}

\Auteur{IKNI Layachi}{}

\Encadrant{Pag\'e Vincent}{MCF à l'Université des Antilles}


\maketitle

\newpage
%\section*{Remerciements}
%\bigskip
%\pagestyle{plain}
%\bigskip
%\newpage 

\tableofcontents
\newpage
\chapter{Introduction }

\section{Installation}
TensorFlow est une librairie de calcul dédiée à l'apprentissage automatique. On peut l'utiliser avec python, java, C,.... Dans notre cas, nous utiliserons python.


Pour l'installation, nous avons suivi les instructions du tutoriel officiel qui se trouve ici, sans difficultés :

https://www.tensorflow.org/install/


\section{Premiers concepts de TensorFlow}

Tout d'abord, TensorFlow s'appuie sur des concepts de programmation très différents d'une programmation standard python. Pour bien les comprendre, prenons un exemple :


On veut que notre programme prenne une valeur réelle (x), calcule une valeur  $y = W*x+b$ avec $W$ et $b$ des valeurs réelles que notre programme sera appelé a modifier plus tard.

le code correspondant en python est le suivant
\begin{lstlisting}
x = 2
W = 0.3
b = -0.3
y = W*x+b
print(y)
\end{lstlisting}

La sortie de ce programme serait :
\begin{verbatim}
0.3
\end{verbatim}

ici, $W$, $b$ , $x$ et $y$ sont des variables du programmes.
Néanmoins, dans le contexte de notre programme, elles jouent des rôles très différents :
\begin{itemize}
\item $x$ est une entrée 
\item $W$ et $b$ sont des valeurs modifiables
\item $y$ est calculé a partir de $x$, $W$ et $b$
\end{itemize}

La programmation en TensorFlow, met en place cette différence.
\begin{itemize}
\item $x$ sera appelé un \textbf{placeholder} , (en deux mots : une variable dont on promet qu'on lui donnera une valeur au moment du run)
\item $W$ et $b$ seront définis comme des variables
\item $y$ sera défini implicitement par l'équation de calcul 
\end{itemize}

Notons qu'il existe aussi la notion de constante, non présentée ici, mais facile a appréhender.

Le code correspondant en TensorFlow est le suivant :
\begin{lstlisting}
import tensorflow as tf

x = tf.placeholder(tf.float32)

W = tf.Variable([.3], dtype=tf.float32)
b = tf.Variable([-.3], dtype=tf.float32)

y = W*x + b

print(y)
\end{lstlisting}

La sortie de ce programme est alors surprenante :
\begin{verbatim}
Tensor("add:0", dtype=float32)
\end{verbatim}

De fait, nous n'avons pas calculé la valeur de $y$.\\
En fait notre programme ne manipule pas des variables au sens traditionnel, mais explique les dépendances entres les différents éléments de notre programme (ce sont des nœuds du graphes de calcul). TensorFlow s'appuie sur ce graphe de calcul, sur lequel nous reviendrons plus tard pour comprendre son intérêt.\\

Le graphe correspondant est représenté ci-dessous pour information.

\chapter*{Conclusion}
Ce T.E.R....... 
\newpage
\begin{thebibliography}{9}
\bibitem{seco}
Hédy Attouch, Paul-\'Emile Maingé, Patrick Redont, \emph{A second-order differential system with hessian-driven damping; application to non-elastic shock laws}, 2013
\bibitem{attou}
Hédy Attouch, \emph{Variational Convergence for Functions and Operators}, 1986
\bibitem{Brezis}
Haim Brézis, Analyse fontionnelle, 1987
\bibitem{Brezis2}
Haim Brézis, \emph{Opérateurs maximaux monotones et semie-groupes de contractions dans les espaces de Hilbert}, 1973
\bibitem{Nawfal}
Nawfal El Hage Hassan, \emph{Topologie générale et espaces normés}, 2011
\end{thebibliography}
\addcontentsline{toc}{chapter}{Références}
\bibliographystyle{alpha}

\end{document}
